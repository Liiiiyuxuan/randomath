\documentclass{article}
% \synctex=1
% \newcommand{\fix}[1]{{\bf *** #1 ***}}
\usepackage{amsfonts, amsmath, comment, hyperref, fontawesome5}
\usepackage{mathrsfs, amssymb, tikz-cd, booktabs}
\usepackage{stmaryrd, siunitx, lmodern, multicol, adjustbox}
\usepackage{multirow, pifont, soul, enumerate}
\usepackage{latexsym, cancel, tikz, xcolor, url, color}
\usepackage{listings, amsthm}
\usepackage{imakeidx}
\usepackage[ruled,linesnumbered,boxruled,slide]{algorithm2e}
\usepackage[symbol]{footmisc}
\usepackage{textcomp}
\usepackage[T1]{fontenc}
\usepackage{marvosym}
\usepackage{harmony}
\usepackage{hieroglf}
\usepackage[noend]{algpseudocode}
\usepackage{colortbl}
\usepackage{wasysym}

\hypersetup{
    colorlinks,
    linkcolor={red!50!black},
    citecolor={blue!50!black},
    urlcolor={blue!80!black}
}

\usepackage[OT2,OT1]{fontenc}
\def\SH{\mbox{\fontencoding{OT2}\selectfont\char88}}

\newcommand\blfootnote[1]{%
    \begingroup
    \renewcommand\thefootnote{}\footnote{#1}%
    \addtocounter{footnote}{-1}%
    \endgroup
}

\linespread{1.2} 

\usepackage[all]{xy}
\usetikzlibrary{calc}
\usetikzlibrary{shapes, positioning}
\tikzset{
	arr/.style={-stealth,shorten >=4.2mm,shorten <=4.2mm,thick}, %
	dot/.style={rotate=-45,font=\LARGE}, %
	dot2/.style={rotate=45,font=\LARGE} %
}

\usepackage{amssymb}
\usepackage[many]{tcolorbox}
\newtcolorbox[auto counter,number within=section]{Question}[2][]{%
    colback=green!5,
    colframe=green!35!black,
    colbacktitle=green!35!black,
    coltitle=white,
    fonttitle=\bfseries, 
    title=Question~\thetcbcounter.\ #2,
    enhanced,
    attach boxed title to top left={yshift=-2mm, xshift=0.5cm},%
    #1% 
}

\lstset{
    basicstyle=\ttfamily, % Set the default font for listings to typewriter
    mathescape=true,      % Allows escaping to LaTeX math mode within $$
    columns=fullflexible,
    breaklines=true,      % Set automatic line breaking
    captionpos=b,         % Sets the caption-position to bottom
    xleftmargin=\parindent,
    numbers=left,         % Line numbers on left
    numberstyle=\small,   % Line numbers styling
    numbersep=5pt,
    escapeinside={(*@}{@*)} % for escaping to LaTeX inside your code
}

\setlength{\textheight}{8.75in}
\setlength{\textwidth}{6.5in}
\setlength{\topmargin}{0.0in}
\setlength{\headheight}{0.0in}
\setlength{\headsep}{0.0in}
\setlength{\leftmargin}{0.0in}
\setlength{\oddsidemargin}{0.0in}
\setlength{\parindent}{3pc}

\newenvironment{solution}[1][\proofname]{
    \proof[\textbf{Solution:}] \renewcommand{\qedsymbol}{$\bell$}
}{\endproof}

\begin{document}

\begin{center}
    Solution to Exercises
\end{center} 

% *
% !
% ?
% TODO 
% FIXME: 
% // 

\noindent \textbf{Exercise 3.1}: 

\begin{solution}

\end{solution}

\noindent \textbf{Exercise 5.1}: 

\begin{solution}
    Suppose for a contradiction that $L$ is a regular language, thus there exists pumping length $p$ satisfying the pumping lemma. Now consider the string 
    \[ \omega := 0^p 1^p \in L \]
    of length $2p$. For this string, in its decomposition $\omega = xyz$, we have $xy$ is made of purely zero's. In particular, $y$ consists of only zero's. It is now easy to see that 
    \[ x y^i z \notin L \]
    for, for example, $i = 2$. This contradicts our assumption that $L$ is regular, hence $L$ is not regular. 
\end{solution}

\noindent \textbf{Exercise 5.2} \label{exercise5.2}: 

\begin{solution}
    
\end{solution}

\noindent \textbf{Exercise 5.3} \label{exercise5.3}: 

\begin{solution}
    Recall that we know for $\omega = d_m d_{m-1} \cdots d_0$, we have defined 
    \[ [\omega]_k =  \]
\end{solution}

\noindent \textbf{Exercise 5.4} \label{exercise5.4}: 

\begin{solution}
    
\end{solution}

\noindent \textbf{Exercise 5.5} \label{exercise5.5}: 

\begin{solution}
    This is similar to exercise 5.1. Suppose for a contradiction that the set of palindromes is regular, then there exists a pumping length $p$ satisfying the pumping lemma. Now consider an arbitrary string $\omega$ (palindrome) of length $2p + 1$, we know that in the decomposition $\omega xyz$, $xy$ does not contain the middle character of the palindrome $\omega$. As a result, it is now easy to see that the new string 
    \[ x y^i z \]
    wouldn't be a palindrome as desired, which further implies that $x y^i z \notin L$. This contradicts the pumping lemma, which means that that $L$ is not regular. 
\end{solution}




\end{document}